\documentclass[10pt]{article}
\usepackage[margin=1in]{geometry}
\usepackage{multicol}
\usepackage{fancyhdr}
\usepackage{titlesec}
\usepackage{lipsum} % for dummy text; remove when not needed
\usepackage{parskip} % optional: spacing between paragraphs

\setlength{\columnsep}{1cm} % space between columns
\pagestyle{fancy}
\fancyhf{}
\rhead{\today}
\lhead{Draft notes}
\cfoot{\thepage}

\titleformat{\section}{\bfseries\large}{}{0em}{}[\titlerule]

\begin{document}
	
	\section*{Probabilities}
	
	% Begin Cornell layout
	\begin{minipage}[t]{0.4\textwidth}
		%Left side
		\textbf{}
		\vspace{0.3cm}
		\textbf{Axioms of Probability}
		\begin{enumerate}
			\item For any event $A$, $0 \leq P(A) \geq 1$.
			\item $\Omega \in \mathcal{F} \Rightarrow P(\Omega) = 1$
			\item $P(\cup_i A_i) = \sum_i P(A_i)$
		\end{enumerate}
		\vspace{0.3cm}
		Where $\mathcal{F}$ is the \textit{domain} of the probability measue $P$
		 
		 
	\end{minipage}
	\hfill
	\begin{minipage}[t]{0.58\textwidth}
		%Right side
		\textbf{Introduction}
		\vspace{0.3cm}\\
		
\begin{itemize}
	\item Experiment : Process whose outcome is not know in advance.
	\item Sample space : Set $\Omega$ of all \textit{possible} outcomes from the experiment.
	\item Event : Subset of $\Omega$ . \textit{Statement} about the outcome of an experiment. 
\end{itemize}
\vspace{0.3cm}
\textit{Example} : When rolling 2 dice :
There are  $6^2=36$ outcomes $\Omega =\{(m,n):1\leq m,n \geq 6\}$ . If we state that only '\textit{the sum is 9}' then : $B=\{(6,3),(5,4),(4,5),(3,6)\}$
\vspace{0.3cm}\\
\textbf{Additivity}
\vspace{0.3cm}\\
 If $A_1, A_2, A_3,...$ are \textbf{disjointed} events, that is \textit{two events cannot happen at the same time} (i.e. $A_i \cap A_j = \empty$ for all $i \neq j$)	then :
$$P\left( \bigcup_{i=1}^{\infty} A_i \right) = \sum_{i=1}^{\infty} P(A_i)$$\\
 The addition in \textit{disjoint} means calculates the \textbf{total chance that one or the other} happens.
 


	\end{minipage}
	
	\vspace{1cm}
	\noindent\textbf{Summary:}
\end{document}
